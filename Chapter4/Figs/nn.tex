\documentclass[border=0mm]{standalone}

\usepackage{bm}
\usepackage{tikz}
\usepackage{amsmath}
\usepackage{amssymb}

\newcommand{\vx}{{x}}
\newcommand{\vy}{\bm{y}}
\newcommand{\vz}{\bm{z}}
\newcommand{\vh}{\bm{h}}
\newcommand{\MW}{\bm{w}}
\newcommand{\MV}{\bm{v}}
\newcommand{\trans}{^{\top}}

\begin{document}
\pagestyle{empty}

\def\layersep{1.2cm}
\def\Xdim{3}
\def\Zonedim{11}
\def\Honedim{2}
\def\Ztwodim{11}
\def\Htwodim{3}
\def\Zthreedim{11}
\def\Ydim{1}

\begin{tikzpicture}[shorten >=1pt,->,draw=black!20, node distance=\layersep]
    \tikzstyle{every pin edge}=[<-,shorten <=1pt]
    \tikzstyle{neuron}=[circle,minimum size=10pt,inner sep=0pt, draw=black]
    \tikzstyle{annot} = [right, font=\large]

    % Inputs
    \foreach \name / \x in {1,...,\Xdim}
    % This is the same as writing \foreach \name / \y in {1/1,2/2,3/3,4/4}
        \node[neuron] (X-\name) at (\x,0) {};

    % z_1
    \foreach \name / \x in {1,...,\Zonedim}
        \path[xshift=-4cm]
            node[neuron] (Z1-\name) at (\x, -\layersep) {};

    % h_1
    % \foreach \name / \x in {1,...,\Honedim}
    %     \path[xshift=.55cm]
    %         node[neuron] (H1-\name) at (\x, -2 * \layersep) {};

    % z_2
    \foreach \name / \x in {1,...,\Ztwodim}
        \path[xshift=-4cm]
            node[neuron] (Z2-\name) at (\x, -3 * \layersep) {};

    % h_2
    % \foreach \name / \x in {1,...,\Htwodim}
    %     \path node[neuron] (H2-\name) at (\x, -4 * \layersep) {};

    % z_3
    \foreach \name / \x in {1,...,\Zthreedim}
        \path[xshift=-4cm]
            node[neuron] (Z3-\name) at (\x, -5 * \layersep) {};

    % % y
    \foreach \name / \x in {1,...,\Ydim}
        \path[xshift=1cm]
            node[neuron] (Y-\name) at (\x, -6 * \layersep) {};

    % Draw the output layer node
    % \node[output neuron,pin={[pin edge={->}]right:Output}, right of=H-3] (O) {};

    % Connect every node in the input layer with every node in the
    % hidden layer.
    \foreach \source in {1,...,\Xdim}
        \foreach \dest in {1,...,\Zonedim}
            \path (X-\source) edge (Z1-\dest);

    \foreach \source in {1,...,\Zonedim}
        \foreach \dest in {1,...,\Ztwodim}
            \path (Z1-\source) edge (Z2-\dest);

    \foreach \source in {1,...,\Ztwodim}
        \foreach \dest in {1,...,\Zthreedim}
            \path (Z2-\source) edge (Z3-\dest);

    \foreach \source in {1,...,\Zthreedim}
        \foreach \dest in {1,...,\Ydim}
            \path (Z3-\source) edge (Y-\dest);

    % % Connect every node in the hidden layer with the output layer
    % \foreach \source in {1,...,5}
    %     \path (H-\source) edge (O);

    % % Annotate the layers
    \node[annot] (x) at (7.3, -0 * \layersep) {$\vx \in\mathbb{R}^d$};
    \node[annot] (x) at (7.3, -1 * \layersep) {$\vz_1 = \sigma(\MW_1 \vx)$};
    \node[annot] (x) at (7.3, -3 * \layersep) {$\vz_2 = \sigma((\MW_2 \MV_1^\top) \vh_1)$};
    \node[annot] (x) at (7.3, -5 * \layersep) {$\vz_3 = \sigma((\MW_3 \MV_2^\top) \vh_2)$};
    \node[annot] (x) at (7.3, -6 * \layersep) {$\vy = \MV_3\trans \vz_3$};


\end{tikzpicture}

\end{document}